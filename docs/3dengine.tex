% COSC 3P98 Project
% gl-billiards
% Dan Lapp
% Taras Mychaskiw

\section{3D Engine}

	\subsection{Loading Models}
	To use 3d models that are more complex than simple primitive shapes, it is important to import custom models. Gl-billiards uses 3D Studio Max to create 
and texture all of the models in the game. The main advantage of this approach is that all vertex coordinates and texture mapping coordinates can be handled 
by 3D Studio and loaded into OpenGL. Programming a pool table directly in OpenGL would be virtually impossible.\\
	Gl-billiards uses the \textit{.3ds} file format to read models into OpenGL. This is a binary file format that is fairly simple to parse, well documented, and 
natively supported by 3D Studio. The binary data is organized into a heirarchy of chunks. An imported model is abstracted into it's own class \textit{Model}. 
Each object in the scene, such as \textit{Table}, and \textit{Cue} have a \textit{Model}. Initialization is shown in Figure \ref{eqn:modelInit}
	\begin{equation}
	\label{eqn:modelInit}
	Model{\enskip}model = Model(std::string{\enskip}modelPath,{\enskip}std::string{\enskip}texturePath)
	\end{equation}
The texture specified must be a \textit{.bmp} file for the texture parser to read. When the model is initialized, the 3ds parser skips through the file chunks 
and stores three things. First are the actual \textit{(x, y, z)} coordinates of the vertices, then the polygon list, which is the order to draw the vertices. Last 
is the texture mapping coordinates.\\
	The bitmap parser reads in a \textit{.bmp} file used as a texture for the model. The texture must have a width and a height that is a power of $2$, 
and must be maximum $2048$ pixels in any dimension. The textures are then handed to OpenGL and a texture id is saved in the model.

	%end{Loading Models}

	\subsection{Calculating Normals}
	After the model and texture is loaded, normals have to be calculated to be used with game lighting. Gl-billiards calculates normals itself. This way 
we could get the vertex normals without having to parse the smoothing groups in the \textit{.3ds} file. Smoothing groups are a different representation 
of smoothed vertex normals used by 3D Studio Max, and are not directly used with OpenGL's \textit{glNormal3f()}. Since every polygon in this file 
format is a triangle, calculating normals was fairly simple. 
The pseudocode for this is shown below in Algorithm \ref{algo:calcNormals}

	\begin{center}
	\begin{pseudocode}[framebox]{calculateNormals}{vertexList, polygonList}
	\label{algo:calcNormals}
	\LOCAL{numPolygons}[vertexList.size]\\
	\FOREACH polygon \in \text{polygonList} \DO
	\BEGIN
	\vec v_1 \gets\vec polygon.P_1 - \vec polygon.P_2\\
	\vec v_2 \gets\vec polygon.P_1  - \vec polygon.P_3 \\
	\vec n \gets \vec v_1 \times \vec v_2 \\\\
		\FOREACH coordinate \in \text{i} \DO
		\BEGIN
			\text{add } \vec n \text{ to normalList[i.coordinate]}\\
			\text{numPolygons[i.coordinate]}++
		\END
	\END
	\\\\
	\FOREACH i \in \text{normalList} \DO
		\IF{numPolygons[i] > 1}
		\THEN{i (1.0 / numPolygons[i])}
    \end{pseudocode}
    \end{center}
	
	First, for each triangular polygon \textit{$(P_1, P_2, P_3)$} polygonList, calculate two vectors $\vec v_1, \vec v_2$ to use for the cross
 product. The cross product returns a vector $\vec n$ perpendicular to the polygon face. \textit{numPolygons} keeps track of which surfaces are next 
to each vertex. Then in the final \textit{for} loop the polygon normals are averaged to give each vertex normal a smoothed effect.

	%\end{Calculating Normals}

	\subsection{Optimizations}
	To optimize 3d rendering, OpenGL display lists are implemented for \textit{Model}. Large floating point arrays used to store vertex coordinates 
require alot of memory to manage these arrays. Using display lists significantly reduces the overhead associated with storing vertex arrays. 
From the results of running benchmarks on gl-billiards, the user interface used $\sim43\%$ of the program's computation time, and
 physics used $\sim27\%$. Therefore speeding up the model rendering provided a noticable speed increase of at least $5\times$.

	%\end{Optimizations}
	

	


%\end{3D Engine}