\documentclass[12pt]{article}
\usepackage{graphicx}
\usepackage{float}
\usepackage{amsmath}
\usepackage{scrextend}
\addtolength{\textwidth}{1in}
\addtolength{\textheight}{1in}
\addtolength{\evensidemargin}{0.5in}
\addtolength{\oddsidemargin}{-0.5in}
\addtolength{\topmargin}{-0.5in}
\title{gl-billiards}
\author{Dan Lapp\\Taras Mychaskiw\\COSC 3P98 Project}
\date{January 8th, 2014}

\begin{document}
\maketitle
    \begin{abstract}
    \end{abstract}

    \thispagestyle{empty}
    \tableofcontents
    %\cleardoublepage
    %\newpage
    \thispagestyle{empty}
    \mbox{}
    \clearpage
    \setcounter{page}{1}
    
    \section{Physics Engine}
    
        \subsection{Taking a Shot}
        Once a ball is hit by the cue, there are two ways energy is transferred to the ball. First, and more noticeable of the two
        is the transfer of linear momentum. This sends the ball moving around the table in straight lines. Second is the transfer
        of angular momentum, which applies spin to the ball. Spin can cause the ball to stop in it's tracks after a collision,
        bounce off cushions in strange (but predictable) ways and to any pool player, is just as if not more important an aspect
        to the game as the linear momentum.
            \subsubsection{Linear Momentum}
            To simplify the shot making process, the pool cue is assuming to act like a spring. That is, the further you pull
            the cue away from the cue you are going to hit, the more powerful your shot will be. Spring motion can be described
            using Hooke's Law,
            \begin{equation}
                \vec F = -k \vec x
                \label{equ:hookes_law}
            \end{equation}
            where, $\vec F$ is the force, $\vec x$ is how much the spring is stretched and $k$ is the spring constant, which
            describes basically how powerful (springy) the spring is.
            \\
            Hooke's Law is equivalent to Newton's Second Law, which states
            \begin{equation}
                \vec F = m \vec a
                \label{equ:newton}
            \end{equation}
            where, $\vec F$ is the force, $m$ is the mass of the object and $\vec a$ is the acceleration of the object. Setting
            Equation~\ref{equ:hookes_law} equal to Equation~\ref{equ:newton} gives
            \begin{eqnarray}
                -k \vec x &=& m \vec a \nonumber \\
                -k \vec x &=& m \frac{\Delta \vec v}{\Delta t} \nonumber \\
                \Delta \vec v &=& \frac{-k \Delta t}{m}\vec x  \nonumber
            \end{eqnarray}
            where $\vec v$ is the velocity of the object, and $t$ is the time. For pool, the initial velocity of the ball being struck is
            always zero, so we can simplify the above equation to
            \begin{equation}
                \vec v = \frac{-k t}{m}\vec x
                \label{equ:shot_velocity}
            \end{equation}
            where $\vec x$ is the distance between the ball and the tip of the cue, $m$ is the mass of the ball, $t$ is how much time the
            cue is in contact with the ball and $k$ is the spring constant of the cue. The values for $t$ and $k$ needed to be experimentally
            found though trial and error, essentially trying values until what happened in the simulation closely represented real life. The
            values for $t$ and $k$ were found to be $0.1$s and $10$N/m.
            %\end{Linear Velocity}
            
            \subsubsection{Angular Momentum}
            Angular momentum transfer is more complex. The starting point for this section is also Newton's Second Law, but written
            in a form more usable to the problem at hand.
            \begin{equation}
                \vec \tau = I \vec \alpha
                \label{equ:newton_rotation}
            \end{equation}
            where $\vec \tau$ is the torque applied to the object around it's center of rotation, $I$ is the moment of inertia of the
            object and $\vec \alpha$ is the angular acceleration of the object.
            \\
            Torque is always given by the equation,
            \begin{equation}
                \vec \tau = \vec r \times \vec F
                \label{equ:torque}
            \end{equation}
            where $\vec F$ is the force being applied, $\vec r$ is the vector between where the force is being applied and the center
            of rotation of the object, and $\times$ is the vector cross product.
            \\
            An important note is the inertia for a solid sphere is given by,
            \begin{equation}
                I_{sphere} = \frac{2}{5} m R^2
                \label{equ:inertia_sphere}
            \end{equation}
            where $R$ is the radius and $m$ is the mass of the sphere.
            \\
            Setting Equation~\ref{equ:newton_rotation} equal to Equation~\ref{equ:torque}, subbing in Equation~\ref{equ:inertia_sphere}
            for $I$ and Equation~\ref{equ:hookes_law} for $\vec F$, and solving gives,
            \begin{eqnarray}
                \vec r \times -k \vec x &=& \frac{2}{5} m R^2 \frac{\Delta \vec \omega}{\Delta t} \nonumber \\
                \Delta \vec \omega &=& \frac{5}{2} \left(\frac{\Delta t}{m R^2}\right) (\vec r \times -k \vec x) \nonumber
            \end{eqnarray}
            where $\vec \omega$ is the angular velocity of the ball, and $t$ $t$ is how much time the cue is in contact with the ball (the same $t$
            as in the previous section). Again, for pool, when the ball is struck it is entirely at rest, so the initial angular velocity is zero.
            The above equation then simplifies to
            \begin{equation}
                \vec \omega = \frac{5}{2} \left(\frac{\Delta t}{m R^2}\right) (\vec r \times -k \vec x)
                \label{equ:shot_angular}
            \end{equation}
            where $\vec x$ is the distance between the ball and the tip of the cue, $m$ is the mass of the ball, $R$ is the radius of the ball,
            $t$ is how much time the cue is in contact with the ball, $k$ is the spring constant of the cue and finally $\vec r$ is the vector
            between the center of the ball and where the cue strikes the ball.
            %\end{Angular Velocity}
        %\end{Taking a Shot}
        
        \subsection{Event System}
        For the purposes of this program, an event is a ball-ball collision, or a ball banking off of the cushion. The way gl-billiards works
        is each time the screen is redisplayed, some physics updater function is called with a parameter $dt$, which is how much time has passed
        since the last frame. The engine simulates the pool game up until a time $dt$ has passed. For example, if $1$s was passed, then it would
        simulate $1$s of play before returning, then the screen would be updated as if $1$s had passed.
        \\
        So, the engine determines which events occur within time $dt$. Event detection then becomes \textit{when} do events happen instead
        of \textit{if} they happened. The entire update function works as follows:\\
        \texttt{
            update($dt$)                                                        \\
            \begin{addmargin}[1em]{0em}
            find earliest event that occurs within time $dt$, call it $event$   \\
            if no such event                                                    \\
                \begin{addmargin}[1em]{0em}
                roll balls until $dt$                                           \\
                \end{addmargin}
            otherwise                                                           \\
                \begin{addmargin}[1em]{0em}
                roll balls until $event$ occurs                                 \\
                handle $event$                                                  \\
                $dt$ <-- $dt$ - $event$.$getTime()$                             \\
                call update($dt$) again                                         \\
                \end{addmargin}
            \end{addmargin}
        }
        In this manner, all events that happen within $dt$ and handled automatically before the frame updates and the balls slow
        down due to friction. Note that "roll balls" simply means each ball moves at it's speed for some amount of time. Pocketing
        detection needs to be done there, as it is not a standard event. That is, if a ball is pocketed, it cannot possible participate
        in any events in time $dt$, since it would bank off the cushion or collide with another ball first. \\
        After $update(dt)$ is finished, each ball gets affected by friction and slows down slightly. Friction takes the form of a
        slight decrease in the velocity of the ball. 
        %\end{Event System}
        
        \subsection{Event Detection}
            \subsubsection{Ball-Ball Collision Detection}
            A ball-ball collision will occur when the centers of the balls are two radii apart. In gl-billiards, the trajectory
            of each ball is simplified to be a straight line per frame since friction is only applied at the end of each frame.
            Thus, the trajectory of each ball becomes,
            \begin{equation}
                \vec r = \vec r_0 + \vec v t
                \label{equ:position}
            \end{equation}
            where $\vec r$ is the position of the ball, $\vec r_0$ is the starting position, $\vec v$ is the velocity of the ball and
            $t$ is the amount of time elapsed. Then, for each possible collision to occur, the program needs to find at what times the
            distance from one ball to the other is equal to twice the radius of the ball. Mathematically, this becomes
            \begin{eqnarray}
                | \vec r_1 - \vec r_2 | &=& 2 R   \nonumber   \\
                \sqrt{ (\vec r_1 - \vec r_2) \cdot (\vec r_1 - \vec r_2) } &=& 2 R  \nonumber
            \end{eqnarray}
            where $R$ is the radius of a ball. Squaring both sides, and subbing in Equation~\ref{equ:position} for both
            $\vec r_1$ and $\vec r_2$ (using $\vec{dr_0} = \vec r_{0_1} - \vec r_{0_2}$ and $\vec{dv} = \vec v_1 - \vec v_2$),
            \begin{eqnarray}
                (\vec{dr_0} \cdot \vec{dr_0}) + (\vec{dr_0} \cdot \vec{dv}) t + (\vec{dv} \cdot \vec{dv}) t^2 &=& 4 R^2
                \label{equ:collision_time}
            \end{eqnarray}
            This is a quadratic equation for $t$ which can be solved easily. If there are real roots to the equation, then
            a collision occurs, and it occurs at the minimal possible $t$ value which is a solution to Equation~\ref{equ:collision_time}.
            %\end{Ball-Ball Collision Detection}
            
            \subsubsection{Ball-Wall Collision Detection}
            Ball-wall collision detection is similar to ball-ball, except the ball is a distance one radius away from the wall when it collides.
            Since the wall doesn't move, a different strategy had to be employed to detect when a ball-wall collision occurs. In this case,
            both the and wall positions are described using parametric equations.
            \begin{eqnarray}
                \vec r = \vec r_0 + (dt~\vec v) T_b         \label{equ:ball_parametric} \\
                \vec w = \vec s + (\vec f - \vec s) T_w     \label{equ:wall_parametric}
            \end{eqnarray}
            Where, $\vec r$ is the position of the ball, $\vec r_0$ is the initial position of the ball, $\vec v$ is the velocity of the ball,
            $dt$ is the amount of time that will pass this frame; $\vec w$ is a point on the wall, $\vec s$ is one endpoint on the wall, and $\vec f$
            is the other endpoint of the wall. $T_b$ and $T_w$ are the parametric variables, and range between $0$ and $1$. \\
            To determine when the ball hits the wall, where the ball hits must be determined first. The center of the ball hits the wall when
            $\vec r = \vec w$, thus the ball actually collides with the wall when $\vec r - R \vec n = \vec w$, where $\vec n$ is the normal
            vector to the wall. For simplicity below, we will ignore this offset value, as it is a fairly simple transformation. So, the collision
            occurs when
            \begin{eqnarray}
                r_{0_x} + (dt~v_x) T_b = s_x + (f_x - s_x) T_w  \nonumber \\
                r_{0_y} + (dt~v_y) T_b = s_y + (f_y - s_y) T_w  \nonumber \\
            \end{eqnarray}
            Solving for $T_b$ and $T_w$ gives
            \begin{eqnarray}
                T_b &=& \frac{(f_x - s_x)\cdot(s_y - r_{0_y}) - (f_y - s_y)\cdot(s_y - r_{0_y})}{(dt~v_y)\cdot(f_x - s_x) - (dt~v_x)\cdot(f_y - s_y)} \nonumber \\
                T_w &=& \frac{(dt~v_x)\cdot(s_y - r_{0_y}) - (dt~v_y)\cdot(s_y - r_{0_y})}{(dt~v_y)\cdot(f_x - s_x) - (dt~v_x)\cdot(f_y - s_y)} \nonumber \\
            \end{eqnarray}
            If both $T_b$ and $T_w$ are between $0$ and $1$, then a ball-wall collision occurs. The point at which can be found by subbing in $T_b$ or $T_w$
            into Equations~\ref{equ:ball_parametric} or \ref{equ:wall_parametric} respectfully. Then the amount of time passed can be solved for using Equations~\ref{equ:position}.
            %\end{Ball-Wall Collision Detection}
        %\end{Event Detection}
        
        \subsection{Event Handling and Velocity Updates}
            \subsubsection{Ball-Ball Collisions}
            %\end{Ball-Ball Collisions}
            
            \subsubsection{Ball-Wall Collisions}
            %\end{Ball-Wall Collisions}
            
            \subsubsection{Friction}
            %\end{Friction}
        %\end{Event Handling and Velocity Updates}
    
    %\end{Physics}

\end{document}
