% COSC 3P98 Project
% gl-billiards
% Dan Lapp
% Taras Mychaskiw

\section{Physics Engine}

    \subsection{Taking a Shot}
    Once a ball is hit by the cue, there are two ways energy is transferred to the ball. First, and more noticeable of the two
    is the transfer of linear momentum. This sends the ball moving around the table in straight lines. Second is the transfer
    of angular momentum, which applies spin to the ball. Unfortunately in gl-billiards, spin is not implemented (though some
    time was spent formulating equations for spin). \\
    To simplify the shot making process, the pool cue is assumed to act like a spring. That is, the further you pull
    the cue away from the ball you are going to hit, the more powerful your shot will be. Spring motion can be described
    using Hooke's Law,
    \begin{equation}
        \vec F = -k \vec x
        \label{equ:hookes_law}
    \end{equation}
    where, $\vec F$ is the force, $\vec x$ is how much the spring is stretched and $k$ is the spring constant, which
    describes basically how powerful (springy) the spring is.
    \\
    Hooke's Law is equivalent to Newton's Second Law, which states
    \begin{equation}
        \vec F = m \vec a
        \label{equ:newton}
    \end{equation}
    where, $\vec F$ is the force, $m$ is the mass of the object and $\vec a$ is the acceleration of the object. Setting
    Equation~\ref{equ:hookes_law} equal to Equation~\ref{equ:newton} gives
    \begin{eqnarray}
        -k \vec x &=& m \vec a \nonumber \\
        -k \vec x &=& m \frac{\Delta \vec v}{\Delta t} \nonumber \\
        \Delta \vec v &=& \frac{-k \Delta t}{m}\vec x  \nonumber
    \end{eqnarray}
    where $\vec v$ is the velocity of the object, and $t$ is the time. For pool, the initial velocity of the ball being struck is
    always zero, so we can simplify the above equation to
    \begin{equation}
        \vec v = \frac{-k t}{m}\vec x
        \label{equ:shot_velocity}
    \end{equation}
    where $\vec x$ is the distance between the ball and the tip of the cue, $m$ is the mass of the ball, $t$ is how much time the
    cue is in contact with the ball and $k$ is the spring constant of the cue. The values for $t$ and $k$ needed to be experimentally
    found though trial and error, essentially trying values until what happened in the simulation closely represented real life. The
    values for $t$ and $k$ were found to be $0.01$s and $200$N/m.
    %\end{Taking a Shot}
    
    \subsection{Event System}
    For the purposes of this program, an event is a ball-ball collision, or a ball banking off of the cushion. The way gl-billiards works
    is each time the screen is redisplayed, some physics updater function is called with a parameter $dt$, which is how much time has passed
    since the last frame. The engine simulates the pool game up until a time $dt$ has passed. For example, if $1$s was passed, then it would
    simulate $1$s of play before returning, then the screen would be updated as if $1$s had passed.
    \\
    So, the engine determines which events occur within time $dt$. Event detection then becomes \textit{when} do events happen instead
    of \textit{if} they happened. The entire update function works as described in Algorithm \ref{algo:update} below. \\
    \begin{center}
    \begin{pseudocode}[framebox]{update}{dt}
        event \gets \text{earliest event that occurs with time } dt     \\
        \IF \text{no events}
        \THEN \text{roll balls for } dt                                 \\
        \ELSE \BEGIN
            \text{rolls balls until } event \text{ occurs}              \\
            \text{handle the event, update velocities}                  \\
            \CALL{update}{dt - event.getTime()}
        \END
        \label{algo:update}
    \end{pseudocode}
    \end{center}

    In this manner, all events that happen within $dt$ and handled automatically before the frame updates and the balls slow
    down due to friction. Note that "roll balls" simply means each ball moves at it's speed for some amount of time. Pocketing
    detection needs to be done there, as it is not a standard event. That is, if a ball is pocketed, it cannot possible participate
    in any events in time $dt$, since it would bank off the cushion or collide with another ball first. \\
    After $update(dt)$ is finished, each ball gets affected by friction and slows down slightly. Friction takes the form of a
    slight decrease in the velocity of the ball. 
    %\end{Event System}
    
    \subsection{Event Detection}
        \subsubsection{Ball-Ball Collision Detection}
        A ball-ball collision will occur when the centers of the balls are two radii apart. In gl-billiards, the trajectory
        of each ball is simplified to be a straight line per frame since friction is only applied at the end of each frame.
        Thus, the trajectory of each ball becomes,
        \begin{equation}
            \vec r = \vec r_0 + \vec v t
            \label{equ:position}
        \end{equation}
        where $\vec r$ is the position of the ball, $\vec r_0$ is the starting position, $\vec v$ is the velocity of the ball and
        $t$ is the amount of time elapsed. Then, for each possible collision to occur, the program needs to find at what times the
        distance from one ball to the other is equal to twice the radius of the ball. Mathematically, this becomes
        \begin{eqnarray}
            | \vec r_1 - \vec r_2 | &=& 2 R   \nonumber   \\
            \sqrt{ (\vec r_1 - \vec r_2) \cdot (\vec r_1 - \vec r_2) } &=& 2 R  \nonumber
        \end{eqnarray}
        where $R$ is the radius of a ball. Squaring both sides, and subbing in Equation~\ref{equ:position} for both
        $\vec r_1$ and $\vec r_2$ (using $\vec{dr_0} = \vec r_{0_1} - \vec r_{0_2}$ and $\vec{dv} = \vec v_1 - \vec v_2$),
        \begin{eqnarray}
            (\vec{dr_0} \cdot \vec{dr_0}) + (\vec{dr_0} \cdot \vec{dv}) t + (\vec{dv} \cdot \vec{dv}) t^2 &=& 4 R^2
            \label{equ:collision_time}
        \end{eqnarray}
        This is a quadratic equation for $t$ which can be solved easily. If there are real roots to the equation, then
        a collision occurs, and it occurs at the minimal possible $t$ value which is a solution to Equation~\ref{equ:collision_time}.
        %\end{Ball-Ball Collision Detection}
        
        \subsubsection{Ball-Wall Collision Detection}
        Ball-wall collision detection is similar to ball-ball, except the ball is a distance one radius away from the wall when it collides.
        Since the wall doesn't move, a different strategy had to be employed to detect when a ball-wall collision occurs. In this case,
        both the and wall positions are described using parametric equations.
        \begin{eqnarray}
            \vec r = \vec r_0 + (dt~\vec v) T_b         \label{equ:ball_parametric} \\
            \vec w = \vec s + (\vec f - \vec s) T_w     \label{equ:wall_parametric}
        \end{eqnarray}
        Where, $\vec r$ is the position of the ball, $\vec r_0$ is the initial position of the ball, $\vec v$ is the velocity of the ball,
        $dt$ is the amount of time that will pass this frame; $\vec w$ is a point on the wall, $\vec s$ is one endpoint on the wall, and $\vec f$
        is the other endpoint of the wall. $T_b$ and $T_w$ are the parametric variables, and range between $0$ and $1$. \\
        To determine when the ball hits the wall, where the ball hits must be determined first. The center of the ball hits the wall when
        $\vec r = \vec w$, thus the ball actually collides with the wall when $\vec r - R \vec n = \vec w$, where $\vec n$ is the normal
        vector to the wall. For simplicity below, we will ignore this offset value, as it is a fairly simple transformation. So, the collision
        occurs when
        \begin{eqnarray}
            r_{0_x} + (dt~v_x) T_b = s_x + (f_x - s_x) T_w  \nonumber \\
            r_{0_y} + (dt~v_y) T_b = s_y + (f_y - s_y) T_w  \nonumber
        \end{eqnarray}
        Solving for $T_b$ and $T_w$ gives
        \begin{eqnarray}
            T_b &=& \frac{(f_x - s_x)\cdot(s_y - r_{0_y}) - (f_y - s_y)\cdot(s_y - r_{0_y})}{(dt~v_y)\cdot(f_x - s_x) - (dt~v_x)\cdot(f_y - s_y)} \nonumber \\
            T_w &=& \frac{(dt~v_x)\cdot(s_y - r_{0_y}) - (dt~v_y)\cdot(s_y - r_{0_y})}{(dt~v_y)\cdot(f_x - s_x) - (dt~v_x)\cdot(f_y - s_y)} \nonumber
        \end{eqnarray}
        If both $T_b$ and $T_w$ are between $0$ and $1$, then a ball-wall collision occurs. The point at which can be found by subbing in $T_b$ or $T_w$
        into Equations~\ref{equ:ball_parametric} or \ref{equ:wall_parametric} respectfully. Then the amount of time passed can be solved for using Equations~\ref{equ:position}.
        %\end{Ball-Wall Collision Detection}
    %\end{Event Detection}
    
    \subsection{Event Handling and Velocity Updates}
        \subsubsection{Ball-Ball Collisions}
        The first step in handling a ball-ball collision is to break up the velocity into two components: normal to the collision plane
        and tangential to the plane. The normal components of each ball's velocity are given by
        \begin{eqnarray}
            \vec v_{n_1} &=& (\vec v_1 \cdot \vec n) \vec n       \nonumber \\
            \vec v_{n_2} &=& (\vec v_1 \cdot (-\vec n)) (-\vec n) \nonumber
        \end{eqnarray}
        where $\vec n$ is the normal to the plane. Then, the tangential components can be obtains from
        \begin{eqnarray}
            \vec v_{t_1} &=& \vec v_{n_1} - \vec v_1    \nonumber   \\
            \vec v_{t_2} &=& \vec v_{n_2} - \vec v_2    \nonumber
        \end{eqnarray}
        During the collision, the tangential components of the velocity of each ball do not change. Thus, the collision can be
        reduced to a $1D$ collision along the normal components. For a $1D$ collision, the final velocities are given by
        \begin{eqnarray}
            \vec v_1^\prime &=& \frac{(m_1 - \epsilon m_2) \vec v_1 + (m_2 + \epsilon m_2) \vec v_2}{m_1 + m_2}   \nonumber \\
            \vec v_2^\prime &=& \frac{(m_1 + \epsilon m_1) \vec v_1 + (m_2 - \epsilon m_1) \vec v_2}{m_1 + m_2}   \nonumber
        \end{eqnarray}
        where $\epsilon$ is the coefficient of restitution. In our case $m_1 = m_2$, and $\epsilon = 1$ since all balls have
        an equal mass and the collision is totally elastic. The above equations simplify to
        \begin{eqnarray}
            \vec v_1^\prime &=& \vec v_2    \nonumber   \\
            \vec v_2^\prime &=& \vec v_1    \nonumber
        \end{eqnarray}
        Thus, the normal components of the velocities of the balls simply switch. The final velocities of each ball after the collision are
        \begin{eqnarray}
            \vec v_1^\prime &=& \vec v_{t_1} + \vec v_{n_2} \nonumber   \\
            \vec v_2^\prime &=& \vec v_{t_2} + \vec v_{n_1} \label{equ:after_ball_ball}
        \end{eqnarray}
        %\end{Ball-Ball Collisions}
        
        \subsubsection{Ball-Wall Collisions}
        Ball-wall collisions are much simpler than ball-ball, as the velocity of the ball is reflected through the vector normal to the
        surface of the wall. The velocity after the event becomes
        \begin{equation}
            \vec v^\prime = \vec v - 2 (\vec v \cdot \vec n) \vec n
            \label{equ:after_ball_wall}
        \end{equation}
        However, there is some energy loss due to the rubberiness of the wall. To account for this, the resultant velocity is scaled down by
        a small percentage to represent the loss. In gl-billiards, this speed loss factor is $0.3$, which equates to about a $20\%$ energy
        loss from the collision with the wall.\\
        %\end{Ball-Wall Collisions}
        
        \subsubsection{Friction}
        At the end of each frame, friction slows down every moving ball slightly. There are two types of friction, kinetic and static.
        Kinetic friction slows down objects which are moving, and static friction must be overcome to either start or continue moving.
        Simpler of the two in gl-billiards is static friction, which is just a simple speed check. If the speed of a ball is less than
        some threshold ($1$cm/s in gl-billiards), then the ball's velocity is set to $0$. \\
        Kinetic friction acts in the opposite direction of the ball's velocity, and is equal to
        \begin{equation}
            \vec F_f = -\epsilon m g
            \label{equ:friction}
        \end{equation}
        where $\epsilon$ of the coefficient of kinetic friction between the table surface and a pool ball. Using Equations~\ref{equ:newton},
        this final velocity of the ball can be obtained after the frictional force is applied.
        \begin{equation}
            \vec v^\prime = \vec v + \vec F_f \cdot dt
            \label{equ:friction_velocity}
        \end{equation}
        where $dt$ is the amount of time that passed in the frame. And that's it! That's all of the physics in gl-billiards in a nutshell.
        %\end{Friction}
    %\end{Event Handling and Velocity Updates}
%\end{Physics}
